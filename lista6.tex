\documentclass[12pt,letterpaper]{article}
\usepackage[utf8]{inputenc}
\usepackage{amsmath,amsthm,amsfonts,amssymb,amscd}
\usepackage[table]{xcolor}
\usepackage[margin=2.5cm]{geometry}
\usepackage{ragged2e}
\usepackage{graphicx}
\usepackage{multicol}
\usepackage{gensymb}
\newlength{\tabcont}
\setlength{\parindent}{0.0in}
\setlength{\parskip}{0.05in}

\begin{document}
	
	\large \textbf{Nome}: Luís Felipe de Melo Costa Silva \\
	\textbf{Número USP}: 9297961 
    
	\begin{center}
		\LARGE \bf
		Lista de Exercícios 6 - AGA0215
	\end{center}
	
	\part{}
	
	\begin{multicols}{7}
		\begin{enumerate}
			\item F.
			\item V.
			\item V.
			\item F.
			\item V.
			\item F.
			\item V.
			\item V.
			\item V.
			\item F.
			\item F.
			\item V.
		    \item F.
		    \item V.
		\end{enumerate}
	\end{multicols}
	
	\part{}
		\begin{multicols}{2}
			\begin{enumerate}
				\item Aproximadamente 67\%.
				\item Maior.
				\item Maior/Efeito Estufa.
				\item Reflexão.
				\item Norte.
				\item Ciclone.
				\item Europa e Ganimedes/Júpiter.
				\item Titã/Saturno.
				\item Ganimedes/Júpiter.
				\item Saturno.
				\item Ferro e Níquel/Hidrogênio e Hélio.
				\item Gelo.
			\end{enumerate}
		\end{multicols}
	
	\part{}
		\textbf{1.} Sabemos que $\delta = arctg (\frac{d}{R})$, sendo que $d$ é o diâmetro do objeto de estudo em UA e R é a distância entre eles em UA. Como estamos tratando de distâncias astronômicas, podemos aproximar $\delta$ como $\delta = \frac{d}{R}$. Calculando as distâncias do Sol de Mercúrio no Periélio e no Afélio, temos: \\
		\begin{center}
		$R_p = a\cdot (1-e) = 0,387 \cdot (1-0,206) = 0,307278$ \\
		$R_a = a\cdot (1+e) = 0,387 \cdot (1+0,206) = 0,466722.$ 
		\end{center}
		Então:
		\begin{center}
		$\delta_a = \frac{0,009304}{0,466722} = 0,019934$ rad = $ 1,142134\degree.$ \\
		$\delta_p = \frac{0,009304}{0,307278} = 0,030278$ rad = $ 1,734801\degree.$
		\end{center} 
		
		\textbf{2.}  Primeiro, calculamos a velocidade molecular no nitrogênio em Mercúrio, sob a maior temperatura em Vênus: 
		\begin{center}
		$v_N = 0,157 \cdot \sqrt\frac{700}{28}$ = $0,785$ km/s.
		\end{center} 
	    De acordo com o enunciado, para que o Nitrogênio não escape da atmosfera, sua velocidade molecular não pode exceder $\frac{1}{6}$ da velocidade de escape, ou seja:
	    \begin{center}
	    $v_{esc} > 6 \cdot v_N = 6 \cdot 0,785 = 4,71$ km/s.
	    \end{center}
	    Então, aplicando a fórmula da velocidade de escape: 
	    \begin{center}
	    $v_{esc} = 11,2 \cdot \sqrt \frac{m}{0,38} > 4,71$
	    \end{center}  Logo, 
	    \begin{center}
	    $ M_m > (\frac{4,71}{11,2})^2 \cdot 0,38 = 0,0672031 \cdot M_t = 4,016 \cdot 10^{23}$ kg. 
	    \end{center}
	    
	    \textbf{3.} Como a massa da atmosfera marciana é de 
	    $\frac{1}{150} \cdot M_{atm_t}$, e que $95\%$ corresponde ao $CO_2$, a massa desse gás em Marte é:\\
	    \begin{center}
	    $M_{CO_2} = \frac{5\cdot10^{18}}{150} \cdot 0,95 \cong 3,17 \cdot 10^{16}$ kg. 
	    \end{center}
	    Agora, considerando que a calota polar de marte tem forma de um círculo, temos: 
	    \begin{center}
	    $V_{cal} = \pi r^2h = 3,14 \cdot (1,5 \cdot 10^{6})^2 \cdot 1 = 7,065 \cdot 10^{12}$ m$^3$. \\
	    $M_{cal} = V_{cal} * d_{CO_2} = 7,065 \cdot 10^{12} * 1,6 \cdot 10^3 = 1,1304 \cdot 10^{16}$ m$^3$. 
	    \end{center}
	    
	    \textbf{4.} Sabemos, pela Segunda Lei de Newton, que:\\
	    \begin{center}
	    $ F = m\cdot a = m\cdot g = \frac{G\cdot m \cdot M}{r^2} \to g = \frac{G\cdot M}{r^2}$.\\
	    \end{center}
	    De acordo com os slides, a massa e o raio de Júpiter são respectivamente, $318 \cdot M_t$ e $11 \cdot r_t$, então: \\
	    \begin{center}
	    $g_j = \frac{G\cdot M_j}{r_j^2} = \frac{G\cdot 318 \cdot M_t}{(11 \cdot r_t)^2} = \frac{318}{121} \cdot \frac{G\cdot M_t}{r_t^2} \cong 2,6 \cdot g_t$
	    \end{center} 
	    Analogamente, para Urano, com massa e raio, respectivamente, $14 \cdot M_t$ e $4 \cdot r_t$:
	    \begin{center}
	    $g_u = \frac{G\cdot M_u}{r_u^2} = \frac{G\cdot 14 \cdot M_t}{(4 \cdot r_t)^2} = \frac{14}{16} \cdot \frac{G\cdot M_t}{r_t^2} \cong 0,9 \cdot g_t$
	    \end{center}
	    \textbf {5.} Assumindo que Saturno é esférico, e que seu raio é $9,5 \cdot r_t = 9,5 \cdot 6371 = 6,05 \cdot 10^4$ km $ = 6,05 \cdot 10^7$ m, temos que seu volume é:
	    \begin{center}
	    	$V_s = \frac{4 \pi r^3}{3} = \frac{4 \pi (6,05 \cdot 10^7)^3}{3} = 9,271 \cdot 10^{23}$ m$^3$.
	    \end{center}
	    Portanto, sua massa seria:
	    \begin{center}
	    	$m_s = V_s \cdot 0,08 = 7,147 \cdot 10^{22}$ kg.
	    \end{center}
	    O que é $0,0001257 \cong 0,013\%$ da massa real estimada de Saturno e $0,0119 \cong 1,2\%$ da massa da Terra. \\
	    
	    \textbf{6.} De acordo com a tabela, a massa de Tritão é
	    $m_{tr} = 0,292 \cdot 7,4 \cdot 10^{22} = 2,1608 \cdot 10^{22}$ kg $ = 0,0036 \cdot m_t$ e seu raio é $r_{tr} = 1355$ km $= 0,2127 \cdot r_t$.
	    Podemos calcular então, sua velocidade de escape.
	    \begin{center}
	    	 $v_{esc} = 11,2 \cdot \sqrt \frac{0,0036}{0,2127} \cong 1,457$ km/s
	    \end{center}
	    Com isso, podemos calcular a velocidade molecular do Nitrogênio, sabendo que $t_{tr} \cong 37 K$:
	    \begin{center}
		    $v_N = 0,157 \cdot \sqrt\frac{37}{28}$ = $0,1804$ km/s.
	    \end{center}
	    Como podemos ver, a velocidade molecular do Nitrogênio é menor do que a $\frac{1}{6}$ da velocidade de escape de Tritão, por isso, a atmosfera foi retida. 	    
			 
\end{document}