\documentclass[12pt,letterpaper]{article}
\usepackage[utf8]{inputenc}
\usepackage{amsmath,amsthm,amsfonts,amssymb,amscd}
\usepackage[table]{xcolor}
\usepackage[margin=2.5cm]{geometry}
\usepackage{ragged2e}
\usepackage{graphicx}
\usepackage{multicol}
\usepackage{gensymb}
\usepackage{wasysym}
\usepackage[brazil]{babel}
\newlength{\tabcont}
\setlength{\parindent}{0.0in}
\setlength{\parskip}{0.05in}

\begin{document}
	
	\large \textbf{Nome}: Luís Felipe de Melo Costa Silva \\
	\textbf{Número USP}: 9297961 
    
	\begin{center}
		\LARGE \bf
		Lista de Exercícios 12 - AGA0215
	\end{center}
	
	\part{}
	
	\begin{multicols}{5}
		\begin{enumerate}
			\item V
			\item V
			\item F
			\item F
			\item F
			\item F
			\item F
			\item F
			\item V
			\item V
			\item F
			\item F
			\item V
			\item V
			\item F
			\item F
			\item V
			\item F
			\item F
			\item F
		\end{enumerate}
	\end{multicols}
	
	\part{}
		\begin{multicols}{2}
			\begin{enumerate}
				\item Hubble
				\item Maiores
				\item Menos
				\item Elípticas
				\item Teorema do Viral
				\item Rádio/Infravermelho, Raios-X
				\item Não-estelar
				\item Magnitude Absoluta
				\item Distância
				\item Elípticas Anãs
				\item Aglomerado de Virgem
				\item Perpendiculares
				\item São confundidas com estrelas
				\item Radiação Synchrotron
				\item Um disco de acresção
				\item Homogeneidade
				\item Isotropia
				\item $\frac{1}{5}$
				\item Gravidade
				\item Energia escura
				\item 20\% a 30\%
				\item Perpétua
				\item Com Blueshift, Com Redshift
				\item $14 \cdot 10^9$ anos
			\end{enumerate}
		\end{multicols}	
	
	\part{}
		\textbf{1.} O alargamento é dado por $A = 2\cdot \Delta\lambda$ com $\Delta\lambda = \frac{V{rot}}{c}\cdot\lambda$, portanto:
		\begin{center}
			$A = 2\cdot\lambda\cdot\frac{V{rot}}{c}$\\
			$A = 2\cdot656,3\cdot\frac{350}{3\cdot10^5}$\\
			$A = 1,531$ nm
		\end{center} 
		
		\textbf{2.} Com $Z = 5$, temos que a distância($R_{atual}$) é $7950 \cdot 10^6$ pc $= 7,95 \cdot 10^9$ pc. Pela fórmula da cosmologia relativística, temos que $R = \frac{R_{atual}}{6}$. Com isso, $R = 1,325 \cdot 10^9$ pc. Usando $m - M = 5\cdot logD - 5$, temos que:
		
		\begin{center}
			$22 - M = 5\cdot log(1,32\cdot 10^9) - 5$\\
			$22 - M = 40,60$\\
			$M = -40,60 + 22 = -18,60$	
		\end{center}
		
		\textbf{3.} A lei de Hubble é dada por $v = H_0 \cdot d$, onde $v$ é a velocidade de recessão e $d$ é a distância. Portanto, $d = \frac{v}{H_0}$. Com isso:
		
		\begin{multicols}{2}
			\begin{itemize}
				\item Para  $H_0=60 km/s/Mpc$:\\
				$d = \frac{4000}{60} = 66,67$ Mpc\\
				\item Para  $H_0=70 km/s/Mpc$:\\
				$d = \frac{4000}{70} = 57,14$ Mpc\\
				\item Para  $H_0=80 km/s/Mpc$:\\
				$d = \frac{4000}{80} = 50$ Mpc\\
			\end{itemize}
		\end{multicols}
	
		\textbf{4.} O tempo de Hubble é dado por $t_0 = \frac{1}{H_0}$. No entanto, $H_0$ está relacionado com Mpc. Por isso, teremos que dividi-lo por $3.1\cdot 10^{19}$, que é 1 Mpc em km.
		
		\begin{multicols}{2}
			\begin{itemize}
				\item Para  $H_0=60h km/s/Mpc$:\\
				$t_0 = \frac{1}{\frac{60}{3.1\cdot 10^{19}}} \approx 5,167 \cdot 10^{17} s \approx 16 \cdot 10^9$ anos.\\
				\item Para  $H_0=80h km/s/Mpc$:\\
				$t_0 = \frac{1}{\frac{80}{3.1\cdot 10^{19}}} = 3.875 \cdot 10^{17} \approx 12 \cdot 10^9$ anos.\\
			\end{itemize}
		\end{multicols}
		
		\textbf{5.} Nesse modelo, distância e tempo de relacionam assim: $R \propto t^{\frac{2}{3}}$, logo $t \propto R^{\frac{3}{2}}$. Portanto, para $t = 9 \cdot10^9$ anos, a distância é $4,326 \cdot 10^6 \cdot A$, onde $A$ é uma constante. Dobrando a distância, ela será $2 \cdot 4,326 \cdot 10^6 \cdot A = 8,652 \cdot 10^6 \cdot A$. Nessa condição, o tempo será $t \approx 25 \cdot10^9$ anos. \\
				 
		\textbf{6.} Temos a fórmula 
		\begin{center}
			$E = \frac{mv^2}{2}-\frac{GMm}{r}$
		\end{center}
		como ponto de partida. Sabemos, pela lei de Hubble, que $v = H_0\cdot r$, então:
		\begin{center}
			$E = \frac{mH_0^2r^2}{2}-\frac{GMm}{r}$
		\end{center}
		Assumindo a massa de uma região esférica como $M = \frac{4\pi r^3}{3}\cdot \rho_0$, teremos:
		
		\begin{center}
			$E = \frac{mH_0^2r^2}{2}-\frac{4\pi Gmr^2\rho_0}{3}$
		\end{center}
		
		Com $\rho_c$ definida como $\rho_c = \frac{3H_0^2}{8\pi G}$, então $H_0^2 = \frac{8\pi G\rho_c}{3}$. Logo:
		
		\begin{center}
			$E = \frac{4\pi Gmr^2\rho_c}{3}-\frac{4\pi Gmr^2\rho_0}{3} = \frac{4\pi Gmr^2}{3} \rho_c - \rho_0$
		\end{center}
		
		Então, se $\rho_c > \rho_0$, $E > 0$ e $\rho_c < \rho_0$, $E < 0$
		
\end{document}