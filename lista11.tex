\documentclass[12pt,letterpaper]{article}
\usepackage[utf8]{inputenc}
\usepackage{amsmath,amsthm,amsfonts,amssymb,amscd}
\usepackage[table]{xcolor}
\usepackage[margin=2.5cm]{geometry}
\usepackage{ragged2e}
\usepackage{graphicx}
\usepackage{multicol}
\usepackage{gensymb}
\usepackage{wasysym}
%\usepackage[brazil]{babel}
\newlength{\tabcont}
\setlength{\parindent}{0.0in}
\setlength{\parskip}{0.05in}

\begin{document}
	
	\large \textbf{Nome}: Luís Felipe de Melo Costa Silva \\
	\textbf{Número USP}: 9297961 
    
	\begin{center}
		\LARGE \bf
		Lista de Exercícios 11 - AGA0215
	\end{center}
	
	\part{}
	
	\begin{multicols}{7}
		\begin{enumerate}
			\item F.
			\item V.
			\item F.
			\item V.
			\item V.
			\item F.
			\item V.
			\item F.
			\item F.
			\item V.
			\item V.
			\item V.
		    \item F.
		    \item V.
		\end{enumerate}
	\end{multicols}
	
	\part{}
		\begin{multicols}{2}
			\begin{enumerate}
				\item Estamos dentro dela.
				\item Disco.
				\item Halo.
				\item Brilho.
				\item Faixa de instabilidade.
				\item Maior.
				\item Ordenadas.
				\item Disco.
				\item Randômicas.
				\item Na sua periferia.
				\item Buraco Negro.
				\item Matéria escura.
			\end{enumerate}
		\end{multicols}
	
	\part{}
		\textbf{1.} Vamos usar que $D = 100$ kpc $= 10^6$ pc, $L_{cef} = 30000\cdot L_{\astrosun}$, e que $M_{\astrosun} = +5$. Logo, a magnitude aparente de uma Cefeida é, usando $M = cte -2,5 log Fr$:
        \begin{center}
        	$M = M_{\astrosun} -2,5 \cdot log(\frac{L_{cef}}{L_{\astrosun}}) = 5 - 11,19 = -6,19$
        \end{center}
        
        Para descobrir a máxima distância, temos que achar a máxima magnitude aparente que o HST pode enxergar, portanto, usando $m - M = 5\cdot log D - 5$:
        
        \begin{center}
        	$m - 5 = 5\cdot log 10^5 -5$\\
        	$m = 5\cdot log 10^5$\\
        	$m = 25$
        \end{center}
        
        Agora, podemos estimar a distância, ainda usando essa relação:
        
        \begin{center}
        	$25 - (-6,2) = 5\cdot log D -5$\\
        	$log D = 7,24$\\
        	$D = 17,378 \cdot 10^6$ pc
        \end{center}
        
        \textbf{2.} Considerando que a estrela é afetada pela extinção, temos que $m = m_0 + A_{\lambda}$. Sabendo que a máxima magnitude aparente que o HST pode enxergar é 25, temos que:
        
        \begin{center}
        	$m - M = 5\cdot log D - 5$\\
        	$m - \frac{2,5D}{1000} - M =  5\cdot log D - 5$, pois a magnitude está medida em kpc\\
        	$5\cdot log D + \frac{2,5D}{1000} = M + 5 - m$ \\
        	$5\cdot log D + \frac{2,5D}{1000} = 25 + 5 +6,2$ \\
        	$5\cdot log D + \frac{2,5D}{1000} = 36,2\to $\\
        	$D \cong 6,8$ kpc
        \end{center}
        
		\textbf{3.} Usando que $P = \frac{2\pi r}{v}$, temos:
		
		\begin{center}
			$P = \frac{2\cdot 3,14\cdot 8000 \cdot 3,08 \cdot 10^{13}}{220}$\\
			$P \cong 7\cdot 10^{15} s \cong 221$ milhões de anos 
		\end{center} 
        
        \textbf{4.} Sabendo que $M(M_{\astrosun}) = \frac{r^3}{P^2}$, temos que:
        
        \begin{center}
        	$M(M_{\astrosun}) = \frac{950^3}{15^2}$\\
        	$M(M_{\astrosun}) \cong 3,81 \cdot M_{\astrosun}$        	
        \end{center}
			 
\end{document}